\documentclass[9pt]{article}
%hide links
\usepackage[hidelinks]{hyperref}
%graphics
\usepackage[dvipsnames,svgnames]{xcolor}
\usepackage[pdftex]{graphicx}
%page size & margins
\usepackage{geometry}
\geometry{
  papersize={105mm,148.5mm},
  left=7mm,
  right=7mm,
  bottom=12mm,
  top=6mm
}
%fancy headers
\usepackage{fancyhdr}
\pagestyle{fancy}
\lhead{}
\chead{}
\rhead{}
\lfoot{}
\cfoot{\thepage}
\rfoot{}
\renewcommand{\headrulewidth}{0pt}
\renewcommand{\footrulewidth}{0pt}

%copyleft symbol
\usepackage{textcomp}

\title{CoffeeScript}
\begin{document}

\begin{titlepage}
\pagestyle{plain}
\begin{center}
\vspace*{0.3\textheight}
\includegraphics{/home/roberto/Documentos/e-dit/coffeescript/logo.png}\\[5pt]
\vfill
\end{center}
\end{titlepage}


\tableofcontents

  CoffeeScript is a little language that compiles into JavaScript. Underneath that awkward Java-esque patina, JavaScript has always had a gorgeous heart. CoffeeScript is an attempt to expose the good parts of JavaScript in a simple way. 


  The golden rule of CoffeeScript is: \emph{``It's just JavaScript''}.
  The code compiles one-to-one into the equivalent JS, and there is no interpretation at runtime. You can use any existing JavaScript library seamlessly from CoffeeScript (and vice-versa). The compiled output is readable and pretty-printed, passes through JavaScript Lint without warnings, will work in every JavaScript runtime, and tends to run as fast or faster than the equivalent handwritten JavaScript. 


 \textbf{Latest Version:}
 1.6.2
 
 
\begin{verbatim}
sudo npm install -g coffee-script
\end{verbatim}


\section{ Overview }

 \emph{CoffeeScript on the left, compiled JavaScript output on the right.}
 
 
\begin{tabular}{p{0.5\textwidth} p{0.5\textwidth}}
\begin{verbatim}
# Assignment:
number   = 42
opposite = true

# Conditions:
number = -42 if opposite

# Functions:
square = (x) -> x * x

# Arrays:
list = [1, 2, 3, 4, 5]

# Objects:
math =
  root:   Math.sqrt
  square: square
  cube:   (x) -> x * square x

# Splats:
race = (winner, runners...) ->
  print winner, runners

# Existence:
alert "I knew it!" if elvis?

# Array comprehensions:
cubes = (math.cube num for num in list)

\end{verbatim}
&
\begin{verbatim}
var cubes, list, math, num, number, opposite, race, square,
  __slice = [].slice;

number = 42;

opposite = true;

if (opposite) {
  number = -42;
}

square = function(x) {
  return x * x;
};

list = [1, 2, 3, 4, 5];

math = {
  root: Math.sqrt,
  square: square,
  cube: function(x) {
    return x * square(x);
  }
};

race = function() {
  var runners, winner;

  winner = arguments[0], runners = 2 <= arguments.length ? __slice.call(arguments, 1) : [];
  return print(winner, runners);
};

if (typeof elvis !== "undefined" && elvis !== null) {
  alert("I knew it!");
}

cubes = (function() {
  var _i, _len, _results;

  _results = [];
  for (_i = 0, _len = list.length; _i < _len; _i++) {
    num = list[_i];
    _results.push(math.cube(num));
  }
  return _results;
})();

\end{verbatim}
\end{tabular}


\section{ Installation }

  The CoffeeScript compiler is itself written in CoffeeScript, using the Jison parser generator. The command-line version of coffee is available as a Node.js utility. The core compiler however, does not depend on Node, and can be run in any JavaScript environment, or in the browser (see ``Try CoffeeScript'', above). 


  To install, first make sure you have a working copy of the latest stable version of Node.js, and npm (the Node Package Manager). You can then install CoffeeScript with npm: 
\begin{verbatim}
npm install -g coffee-script
\end{verbatim}


  (Leave off the -g if you don't wish to install globally.) 


  If you'd prefer to install the latest \textbf{master}
 version of CoffeeScript, you can clone the CoffeeScript source repository from GitHub, or download the source directly. To install the lastest master CoffeeScript compiler with npm: 
\begin{verbatim}
npm install -g \url{http://github.com/jashkenas/coffee-script/tarball/master}
\end{verbatim}


  Or, if you want to install to /usr/local, and don't want to use npm to manage it, open the coffee-script directory and run: 
\begin{verbatim}
sudo bin/cake install
\end{verbatim}
\section{ Usage }


  Once installed, you should have access to the coffee command, which can execute scripts, compile .coffee files into .js, and provide an interactive REPL. The coffee command takes the following options: 


\begin{tabular}{llllllllllllll}
\texttt{-c, --compile} &            Compile a \texttt{.coffee} script into a \texttt{.js} JavaScript file           of the same name.          \\ 

\texttt{-m, --map} &            Generate source maps alongside the compiled JavaScript files. Adds           \texttt{sourceMappingURL} directives to the JavaScript as well.          \\ 

\texttt{-i, --interactive} &            Launch an interactive CoffeeScript session to try short snippets.           Identical to calling \texttt{coffee} with no arguments.          \\ 

\texttt{-o, --output [DIR]} &            Write out all compiled JavaScript files into the specified directory.           Use in conjunction with \texttt{--compile} or \texttt{--watch}.          \\ 

\texttt{-j, --join [FILE]} &            Before compiling, concatenate all scripts together in the order they           were passed, and write them into the specified file.           Useful for building large projects.          \\ 

\texttt{-w, --watch} &            Watch files for changes, rerunning the specified command when any           file is updated.          \\ 

\texttt{-p, --print} &            Instead of writing out the JavaScript as a file, print it           directly to \textbf{stdout}.          \\ 

\texttt{-l, --lint} &            If the \texttt{jsl}           (\href{http://www.javascriptlint.com/}{JavaScript Lint})           command is installed, use it           to check the compilation of a CoffeeScript file. (Handy in           conjunction with 
\\\texttt{--watch})          \\ 

\texttt{-s, --stdio} &            Pipe in CoffeeScript to STDIN and get back JavaScript over STDOUT.           Good for use with processes written in other languages. An example:
\\\texttt{cat src/cake.coffee | coffee -sc} \\ 

\texttt{-e, --eval} &            Compile and print a little snippet of CoffeeScript directly from the           command line. For example:
\\\texttt{coffee -e "console.log num for num in [10..1]"} \\ 

\texttt{-b, --bare} &            Compile the JavaScript without the           \hyperlink{lexical-scope}{top-level function safety wrapper}.          \\ 

\texttt{-t, --tokens} &            Instead of parsing the CoffeeScript, just lex it, and print out the           token stream: \texttt{[IDENTIFIER square] [ASSIGN =] [PARAM\_START (]} ...          \\ 

\texttt{-n, --nodes} &            Instead of compiling the CoffeeScript, just lex and parse it, and print           out the parse tree: 
\begin{verbatim}
Expressions
  Assign
    Value "square"
    Code "x"
      Op *
        Value "x"
        Value "x"\end{verbatim} \\ 

\texttt{--nodejs} &            The \texttt{node} executable has some useful options you can set,           such as
\\\texttt{--debug}, \texttt{--debug-brk}, \texttt{--max-stack-size},           and \texttt{--expose-gc}. Use this flag to forward options directly to Node.js.           To pass multiple flags, use \texttt{--nodejs} multiple times.         
\end{tabular}



 \textbf{Examples:}

\begin{itemize}
\item  Compile a directory tree of .coffee files in src into a parallel tree of .js files in lib:\\ 
coffee --compile --output lib/ src/
\item  Watch a file for changes, and recompile it every time the file is saved:\\ 
coffee --watch --compile experimental.coffee
\item  Concatenate a list of files into a single script:\\ 
coffee --join project.js --compile src/*.coffee
\item  Print out the compiled JS from a one-liner:\\ 
coffee -bpe ``alert i for i in [0..10]''
\item  All together now, watch and recompile an entire project as you work on it:\\ 
coffee -o lib/ -cw src/
\item  Start the CoffeeScript REPL (Ctrl-D to exit, Ctrl-Vfor multi-line):\\ 
coffee

\end{itemize}
\section{ Literate CoffeeScript }


  Besides being used as an ordinary programming language, CoffeeScript may also be written in ``literate'' mode. If you name your file with a .litcoffee extension, you can write it as a Markdown document \^a�� a document that also happens to be executable CoffeeScript code. The compiler will treat any indented blocks (Markdown's way of indicating source code) as code, and ignore the rest as comments. 


  Just for kicks, a little bit of the compiler is currently implemented in this fashion: See it as a document, raw, and properly highlighted in a text editor. 


  I'm fairly excited about this direction for the language, and am looking forward to writing (and more importantly, reading) more programs in this style. More information about Literate CoffeeScript, including an example program, are available in this blog post. 


\section{ Language Reference }


 \emph{ This reference is structured so that it can be read from top to bottom, if you like. Later sections use ideas and syntax previously introduced. Familiarity with JavaScript is assumed. In all of the following examples, the source CoffeeScript is provided on the left, and the direct compilation into JavaScript is on the right. }



 \emph{ Many of the examples can be run (where it makes sense) by pressing the \textbf{run}
 button on the right, and can be loaded into the ``Try CoffeeScript'' console by pressing the \textbf{load}
 button on the left. }



  First, the basics: CoffeeScript uses significant whitespace to delimit blocks of code. You don't need to use semicolons ; to terminate expressions, ending the line will do just as well (although semicolons can still be used to fit multiple expressions onto a single line). Instead of using curly braces \{ \} to surround blocks of code in functions, if-statements, switch, and try/catch, use indentation. 


  You don't need to use parentheses to invoke a function if you're passing arguments. The implicit call wraps forward to the end of the line or block expression.\\ 
console.log sys.inspect object \^a�� console.log(sys.inspect(object));


\section{Functions}

Functions are defined by an optional list of parameters in parentheses, an arrow, and the function body. The empty function looks like this: -$>$
 
 
\begin{tabular}{p{0.5\textwidth} p{0.5\textwidth}}
\begin{verbatim}
square = (x) -> x * x
cube   = (x) -> square(x) * x
\end{verbatim}
&
\begin{verbatim}
var cube, square;

square = function(x) {
  return x * x;
};

cube = function(x) {
  return square(x) * x;
};

\end{verbatim}
\end{tabular}

  Functions may also have default values for arguments. Override the default value by passing a non-null argument. 
  
  
\begin{tabular}{p{0.5\textwidth} p{0.5\textwidth}}
\begin{verbatim}
fill = (container, liquid = "coffee") ->
  "Filling the #{container} with #{liquid}..."
\end{verbatim}
&
\begin{verbatim}
var fill;

fill = function(container, liquid) {
  if (liquid == null) {
    liquid = "coffee";
  }
  return "Filling the " + container + " with " + liquid + "...";
};

\end{verbatim}
\end{tabular}

\section{Objects and Arrays}
 The CoffeeScript literals for objects and arrays look very similar to their JavaScript cousins. When each property is listed on its own line, the commas are optional. Objects may be created using indentation instead of explicit braces, similar to YAML. 


\begin{tabular}{p{0.5\textwidth} p{0.5\textwidth}}
\begin{verbatim}
song = ["do", "re", "mi", "fa", "so"]

singers = {Jagger: "Rock", Elvis: "Roll"}

bitlist = [
  1, 0, 1
  0, 0, 1
  1, 1, 0
]

kids =
  brother:
    name: "Max"
    age:  11
  sister:
    name: "Ida"
    age:  9



\end{verbatim}
&
\begin{verbatim}
var bitlist, kids, singers, song;

song = ["do", "re", "mi", "fa", "so"];

singers = {
  Jagger: "Rock",
  Elvis: "Roll"
};

bitlist = [1, 0, 1, 0, 0, 1, 1, 1, 0];

kids = {
  brother: {
    name: "Max",
    age: 11
  },
  sister: {
    name: "Ida",
    age: 9
  }
};

\end{verbatim}
\end{tabular}


  In JavaScript, you can't use reserved words, like class, as properties of an object, without quoting them as strings. CoffeeScript notices reserved words used as keys in objects and quotes them for you, so you don't have to worry about it (say, when using jQuery). 


\begin{tabular}{p{0.5\textwidth} p{0.5\textwidth}}
\begin{verbatim}
$('.account').attr class: 'active'

log object.class



\end{verbatim}
&
\begin{verbatim}
$('.account').attr({
  "class": 'active'
});

log(object["class"]);

\end{verbatim}
\end{tabular}


\section{Lexical Scoping and Variable Safety}
 The CoffeeScript compiler takes care to make sure that all of your variables are properly declared within lexical scope -- you never need to write var yourself.
 
 
\begin{tabular}{p{0.5\textwidth} p{0.5\textwidth}}  
\begin{verbatim}
outer = 1
changeNumbers = ->
  inner = -1
  outer = 10
inner = changeNumbers()

\end{verbatim}
&
\begin{verbatim}
var changeNumbers, inner, outer;

outer = 1;

changeNumbers = function() {
  var inner;

  inner = -1;
  return outer = 10;
};

inner = changeNumbers();

\end{verbatim}
\end{tabular}

  Notice how all of the variable declarations have been pushed up to the top of the closest scope, the first time they appear. \textbf{outer}
 is not redeclared within the inner function, because it's already in scope; \textbf{inner}
 within the function, on the other hand, should not be able to change the value of the external variable of the same name, and therefore has a declaration of its own. 


  This behavior is effectively identical to Ruby's scope for local variables. Because you don't have direct access to the var keyword, it's impossible to shadow an outer variable on purpose, you may only refer to it. So be careful that you're not reusing the name of an external variable accidentally, if you're writing a deeply nested function. 


  Although suppressed within this documentation for clarity, all CoffeeScript output is wrapped in an anonymous function: (function()\{ ... \})(); This safety wrapper, combined with the automatic generation of the var keyword, make it exceedingly difficult to pollute the global namespace by accident. 


  If you'd like to create top-level variables for other scripts to use, attach them as properties on \textbf{window}
, or on the \textbf{exports}
 object in CommonJS. The \textbf{existential operator}
 (covered below), gives you a reliable way to figure out where to add them; if you're targeting both CommonJS and the browser: exports ? this


\section{If, Else, Unless, and Conditional Assignment}
\subsection|{If/else}
 statements can be written without the use of parentheses and curly brackets. As with functions and other block expressions, multi-line conditionals are delimited by indentation. There's also a handy postfix form, with the if or unless at the end. 


  CoffeeScript can compile \textbf{if}
 statements into JavaScript expressions, using the ternary operator when possible, and closure wrapping otherwise. There is no explicit ternary statement in CoffeeScript \^a�� you simply use a regular \textbf{if}
 statement on a single line. 
 
 
\begin{tabular}{p{0.5\textwidth} p{0.5\textwidth}} 
\begin{verbatim}
mood = greatlyImproved if singing

if happy and knowsIt
  clapsHands()
  chaChaCha()
else
  showIt()

date = if friday then sue else jill

\end{verbatim}
&
\begin{verbatim}
var date, mood;

if (singing) {
  mood = greatlyImproved;
}

if (happy && knowsIt) {
  clapsHands();
  chaChaCha();
} else {
  showIt();
}

date = friday ? sue : jill;

\end{verbatim}
\end{tabular}

\section{Splats...}
 The JavaScript \textbf{arguments object}
 is a useful way to work with functions that accept variable numbers of arguments. CoffeeScript provides splats ..., both for function definition as well as invocation, making variable numbers of arguments a little bit more palatable. 
 
 
\begin{tabular}{p{0.5\textwidth} p{0.5\textwidth}} 
\begin{verbatim}
gold = silver = rest = "unknown"

awardMedals = (first, second, others...) ->
  gold   = first
  silver = second
  rest   = others

contenders = [
  "Michael Phelps"
  "Liu Xiang"
  "Yao Ming"
  "Allyson Felix"
  "Shawn Johnson"
  "Roman Sebrle"
  "Guo Jingjing"
  "Tyson Gay"
  "Asafa Powell"
  "Usain Bolt"
]

awardMedals contenders...

alert "Gold: " + gold
alert "Silver: " + silver
alert "The Field: " + rest



\end{verbatim}
&
\begin{verbatim}
var awardMedals, contenders, gold, rest, silver,
  __slice = [].slice;

gold = silver = rest = "unknown";

awardMedals = function() {
  var first, others, second;

  first = arguments[0], second = arguments[1], others = 3 <= arguments.length ? __slice.call(arguments, 2) : [];
  gold = first;
  silver = second;
  return rest = others;
};

contenders = ["Michael Phelps", "Liu Xiang", "Yao Ming", "Allyson Felix", "Shawn Johnson", "Roman Sebrle", "Guo Jingjing", "Tyson Gay", "Asafa Powell", "Usain Bolt"];

awardMedals.apply(null, contenders);

alert("Gold: " + gold);

alert("Silver: " + silver);

alert("The Field: " + rest);

\end{verbatim}
\end{tabular}


\section{Loops and Comprehensions}
 Most of the loops you'll write in CoffeeScript will be \textbf{comprehensions}
 over arrays, objects, and ranges. Comprehensions replace (and compile into) \textbf{for}
 loops, with optional guard clauses and the value of the current array index. Unlike for loops, array comprehensions are expressions, and can be returned and assigned. 


\begin{tabular}{p{0.5\textwidth} p{0.5\textwidth}}
\begin{verbatim}
# Eat lunch.
eat food for food in ['toast', 'cheese', 'wine']

# Fine five course dining.
courses = ['greens', 'caviar', 'truffles', 'roast', 'cake']
menu i + 1, dish for dish, i in courses

# Health conscious meal.
foods = ['broccoli', 'spinach', 'chocolate']
eat food for food in foods when food isnt 'chocolate'

\end{verbatim}
&
\begin{verbatim}
var courses, dish, food, foods, i, _i, _j, _k, _len, _len1, _len2, _ref;

_ref = ['toast', 'cheese', 'wine'];
for (_i = 0, _len = _ref.length; _i < _len; _i++) {
  food = _ref[_i];
  eat(food);
}

courses = ['greens', 'caviar', 'truffles', 'roast', 'cake'];

for (i = _j = 0, _len1 = courses.length; _j < _len1; i = ++_j) {
  dish = courses[i];
  menu(i + 1, dish);
}

foods = ['broccoli', 'spinach', 'chocolate'];

for (_k = 0, _len2 = foods.length; _k < _len2; _k++) {
  food = foods[_k];
  if (food !== 'chocolate') {
    eat(food);
  }
}

\end{verbatim}
\end{tabular}


  Comprehensions should be able to handle most places where you otherwise would use a loop, \textbf{each}
/\textbf{forEach}
, \textbf{map}
, or \textbf{select}
/\textbf{filter}
, for example: shortNames = (name for name in list when name.length $<$ 5)\\ 
 If you know the start and end of your loop, or would like to step through in fixed-size increments, you can use a range to specify the start and end of your comprehension. 
 
 
\begin{tabular}{p{0.5\textwidth} p{0.5\textwidth}} 
\begin{verbatim}
countdown = (num for num in [10..1])


\end{verbatim}
&
\begin{verbatim}
var countdown, num;

countdown = (function() {
  var _i, _results;

  _results = [];
  for (num = _i = 10; _i >= 1; num = --_i) {
    _results.push(num);
  }
  return _results;
})();

\end{verbatim}
\end{tabular}


  Note how because we are assigning the value of the comprehensions to a variable in the example above, CoffeeScript is collecting the result of each iteration into an array. Sometimes functions end with loops that are intended to run only for their side-effects. Be careful that you're not accidentally returning the results of the comprehension in these cases, by adding a meaningful return value \^a�� like true \^a�� or null, to the bottom of your function. 


  To step through a range comprehension in fixed-size chunks, use by, for example:\\ 
evens = (x for x in [0..10] by 2)


  Comprehensions can also be used to iterate over the keys and values in an object. Use of to signal comprehension over the properties of an object instead of the values in an array. 
  
  
\begin{tabular}{p{0.5\textwidth} p{0.5\textwidth}}  
\begin{verbatim}
yearsOld = max: 10, ida: 9, tim: 11

ages = for child, age of yearsOld
  "#{child} is #{age}"

\end{verbatim}
&
\begin{verbatim}
var age, ages, child, yearsOld;

yearsOld = {
  max: 10,
  ida: 9,
  tim: 11
};

ages = (function() {
  var _results;

  _results = [];
  for (child in yearsOld) {
    age = yearsOld[child];
    _results.push("" + child + " is " + age);
  }
  return _results;
})();

\end{verbatim}
\end{tabular}

  If you would like to iterate over just the keys that are defined on the object itself, by adding a hasOwnProperty check to avoid properties that may be inherited from the prototype, use\\ 
for own key, value of object


  The only low-level loop that CoffeeScript provides is the \textbf{while}
 loop. The main difference from JavaScript is that the \textbf{while}
 loop can be used as an expression, returning an array containing the result of each iteration through the loop. 


\begin{tabular}{p{0.5\textwidth} p{0.5\textwidth}}
\begin{verbatim}
# Econ 101
if this.studyingEconomics
  buy()  while supply > demand
  sell() until supply > demand

# Nursery Rhyme
num = 6
lyrics = while num -= 1
  "#{num} little monkeys, jumping on the bed.
    One fell out and bumped his head."

\end{verbatim}
&
\begin{verbatim}
var lyrics, num;

if (this.studyingEconomics) {
  while (supply > demand) {
    buy();
  }
  while (!(supply > demand)) {
    sell();
  }
}

num = 6;

lyrics = (function() {
  var _results;

  _results = [];
  while (num -= 1) {
    _results.push("" + num + " little monkeys, jumping on the bed.    One fell out and bumped his head.");
  }
  return _results;
})();

\end{verbatim}
\end{tabular}


  For readability, the \textbf{until}
 keyword is equivalent to while not, and the \textbf{loop}
 keyword is equivalent to while true. 


  When using a JavaScript loop to generate functions, it's common to insert a closure wrapper in order to ensure that loop variables are closed over, and all the generated functions don't just share the final values. CoffeeScript provides the do keyword, which immediately invokes a passed function, forwarding any arguments. 
  
  
\begin{tabular}{p{0.5\textwidth} p{0.5\textwidth}}  
\begin{verbatim}
for filename in list
  do (filename) ->
    fs.readFile filename, (err, contents) ->
      compile filename, contents.toString()

\end{verbatim}
&
\begin{verbatim}
var filename, _fn, _i, _len;

_fn = function(filename) {
  return fs.readFile(filename, function(err, contents) {
    return compile(filename, contents.toString());
  });
};
for (_i = 0, _len = list.length; _i < _len; _i++) {
  filename = list[_i];
  _fn(filename);
}

\end{verbatim}
\end{tabular}

\section{Array Slicing and Splicing with Ranges}
 Ranges can also be used to extract slices of arrays. With two dots (3..6), the range is inclusive (3, 4, 5, 6); with three dots (3...6), the range excludes the end (3, 4, 5). Slices indices have useful defaults. An omitted first index defaults to zero and an omitted second index defaults to the size of the array. 


\begin{tabular}{p{0.5\textwidth} p{0.5\textwidth}}
\begin{verbatim}
numbers = [1, 2, 3, 4, 5, 6, 7, 8, 9]

start   = numbers[0..2]

middle  = numbers[3...6]

end     = numbers[6..]

copy    = numbers[..]

\end{verbatim}
&
\begin{verbatim}
var copy, end, middle, numbers, start;

numbers = [1, 2, 3, 4, 5, 6, 7, 8, 9];

start = numbers.slice(0, 3);

middle = numbers.slice(3, 6);

end = numbers.slice(6);

copy = numbers.slice(0);

\end{verbatim}
\end{tabular}


  The same syntax can be used with assignment to replace a segment of an array with new values, splicing it. 


\begin{tabular}{p{0.5\textwidth} p{0.5\textwidth}}
\begin{verbatim}
numbers = [0, 1, 2, 3, 4, 5, 6, 7, 8, 9]

numbers[3..6] = [-3, -4, -5, -6]

\end{verbatim}
&
\begin{verbatim}
var numbers, _ref;

numbers = [0, 1, 2, 3, 4, 5, 6, 7, 8, 9];

[].splice.apply(numbers, [3, 4].concat(_ref = [-3, -4, -5, -6])), _ref;

\end{verbatim}
\end{tabular}


  Note that JavaScript strings are immutable, and can't be spliced. 


\section{Everything is an Expression (at least, as much as possible)}
 You might have noticed how even though we don't add return statements to CoffeeScript functions, they nonetheless return their final value. The CoffeeScript compiler tries to make sure that all statements in the language can be used as expressions. Watch how the return gets pushed down into each possible branch of execution in the function below. 


\begin{tabular}{p{0.5\textwidth} p{0.5\textwidth}}
\begin{verbatim}
grade = (student) ->
  if student.excellentWork
    "A+"
  else if student.okayStuff
    if student.triedHard then "B" else "B-"
  else
    "C"

eldest = if 24 > 21 then "Liz" else "Ike"

\end{verbatim}
&
\begin{verbatim}
var eldest, grade;

grade = function(student) {
  if (student.excellentWork) {
    return "A+";
  } else if (student.okayStuff) {
    if (student.triedHard) {
      return "B";
    } else {
      return "B-";
    }
  } else {
    return "C";
  }
};

eldest = 24 > 21 ? "Liz" : "Ike";

\end{verbatim}
\end{tabular}


  Even though functions will always return their final value, it's both possible and encouraged to return early from a function body writing out the explicit return (return value), when you know that you're done. 


  Because variable declarations occur at the top of scope, assignment can be used within expressions, even for variables that haven't been seen before: 


\begin{tabular}{p{0.5\textwidth} p{0.5\textwidth}}
\begin{verbatim}
six = (one = 1) + (two = 2) + (three = 3)

\end{verbatim}
&
\begin{verbatim}
var one, six, three, two;

six = (one = 1) + (two = 2) + (three = 3);

\end{verbatim}
\end{tabular}

  Things that would otherwise be statements in JavaScript, when used as part of an expression in CoffeeScript, are converted into expressions by wrapping them in a closure. This lets you do useful things, like assign the result of a comprehension to a variable: 
  
  
\begin{tabular}{p{0.5\textwidth} p{0.5\textwidth}}
\begin{verbatim}
# The first ten global properties.

globals = (name for name of window)[0...10]

\end{verbatim}
&
\begin{verbatim}
var globals, name;

globals = ((function() {
  var _results;

  _results = [];
  for (name in window) {
    _results.push(name);
  }
  return _results;
})()).slice(0, 10);

\end{verbatim}
\end{tabular}


  As well as silly things, like passing a \textbf{try/catch}
 statement directly into a function call: 


\begin{tabular}{p{0.5\textwidth} p{0.5\textwidth}}
\begin{verbatim}
alert(
  try
    nonexistent / undefined
  catch error
    "And the error is ... #{error}"
)


\end{verbatim}
&
\begin{verbatim}
var error;

alert((function() {
  try {
    return nonexistent / void 0;
  } catch (_error) {
    error = _error;
    return "And the error is ... " + error;
  }
})());

\end{verbatim}
\end{tabular}

  There are a handful of statements in JavaScript that can't be meaningfully converted into expressions, namely break, continue, and return. If you make use of them within a block of code, CoffeeScript won't try to perform the conversion. 


\section{Operators and Aliases}
 Because the == operator frequently causes undesirable coercion, is intransitive, and has a different meaning than in other languages, CoffeeScript compiles == into ===, and != into !==. In addition, is compiles into ===, and isnt into !==. 


  You can use not as an alias for !. 


  For logic, and compiles to \&\&, and or into ||. 


  Instead of a newline or semicolon, then can be used to separate conditions from expressions, in \textbf{while}
, \textbf{if}
/\textbf{else}
, and \textbf{switch}
/\textbf{when}
 statements. 


  As in YAML, on and yes are the same as boolean true, while off and no are boolean false. 


 unless can be used as the inverse of if. 


  As a shortcut for this.property, you can use @property. 


  You can use in to test for array presence, and of to test for JavaScript object-key presence. 


  All together now: 


\begin{tabular}{ccccccccccc}
CoffeeScriptJavaScript &is=== &isnt!== &not! &and\&\& &or|| &true, yes, ontrue &false, no, offfalse &@, thisthis &ofin &in\emph{no JS equivalent}
\end{tabular}


\begin{tabular}{p{0.5\textwidth} p{0.5\textwidth}}
\begin{verbatim}
launch() if ignition is on

volume = 10 if band isnt SpinalTap

letTheWildRumpusBegin() unless answer is no

if car.speed < limit then accelerate()

winner = yes if pick in [47, 92, 13]

print inspect "My name is #{@name}"

\end{verbatim}
&
\begin{verbatim}
var volume, winner;

if (ignition === true) {
  launch();
}

if (band !== SpinalTap) {
  volume = 10;
}

if (answer !== false) {
  letTheWildRumpusBegin();
}

if (car.speed < limit) {
  accelerate();
}

if (pick === 47 || pick === 92 || pick === 13) {
  winner = true;
}

print(inspect("My name is " + this.name));

\end{verbatim}
\end{tabular}


\section{The Existential Operator}
 It's a little difficult to check for the existence of a variable in JavaScript. if (variable) ... comes close, but fails for zero, the empty string, and false. CoffeeScript's existential operator ? returns true unless a variable is \textbf{null}
 or \textbf{undefined}
, which makes it analogous to Ruby's nil?


  It can also be used for safer conditional assignment than ||= provides, for cases where you may be handling numbers or strings. 


\begin{tabular}{p{0.5\textwidth} p{0.5\textwidth}}
\begin{verbatim}
solipsism = true if mind? and not world?

speed = 0
speed ?= 15

footprints = yeti ? "bear"

\end{verbatim}
&
\begin{verbatim}
var footprints, solipsism, speed;

if ((typeof mind !== "undefined" && mind !== null) && (typeof world === "undefined" || world === null)) {
  solipsism = true;
}

speed = 0;

if (speed == null) {
  speed = 15;
}

footprints = typeof yeti !== "undefined" && yeti !== null ? yeti : "bear";

\end{verbatim}
\end{tabular}

  The accessor variant of the existential operator ?. can be used to soak up null references in a chain of properties. Use it instead of the dot accessor . in cases where the base value may be \textbf{null}
 or \textbf{undefined}
. If all of the properties exist then you'll get the expected result, if the chain is broken, \textbf{undefined}
 is returned instead of the \textbf{TypeError}
 that would be raised otherwise. 


\begin{tabular}{p{0.5\textwidth} p{0.5\textwidth}}
\begin{verbatim}
zip = lottery.drawWinner?().address?.zipcode

\end{verbatim}
&
\begin{verbatim}
var zip, _ref;

zip = typeof lottery.drawWinner === "function" ? (_ref = lottery.drawWinner().address) != null ? _ref.zipcode : void 0 : void 0;

\end{verbatim}
\end{tabular}

  Soaking up nulls is similar to Ruby's andand gem, and to the safe navigation operator in Groovy. 


\section{Classes, Inheritance, and Super}
 JavaScript's prototypal inheritance has always been a bit of a brain-bender, with a whole family tree of libraries that provide a cleaner syntax for classical inheritance on top of JavaScript's prototypes: Base2, Prototype.js, JS.Class, etc. The libraries provide syntactic sugar, but the built-in inheritance would be completely usable if it weren't for a couple of small exceptions: it's awkward to call \textbf{super}
 (the prototype object's implementation of the current function), and it's awkward to correctly set the prototype chain. 


  Instead of repetitively attaching functions to a prototype, CoffeeScript provides a basic class structure that allows you to name your class, set the superclass, assign prototypal properties, and define the constructor, in a single assignable expression. 


  Constructor functions are named, to better support helpful stack traces. In the first class in the example below, this.constructor.name is ``Animal''. 


\begin{tabular}{p{0.5\textwidth} p{0.5\textwidth}}
\begin{verbatim}
class Animal
  constructor: (@name) ->

  move: (meters) ->
    alert @name + " moved #{meters}m."

class Snake extends Animal
  move: ->
    alert "Slithering..."
    super 5

class Horse extends Animal
  move: ->
    alert "Galloping..."
    super 45

sam = new Snake "Sammy the Python"
tom = new Horse "Tommy the Palomino"

sam.move()
tom.move()

\end{verbatim}
&
\begin{verbatim}
var Animal, Horse, Snake, sam, tom, _ref, _ref1,
  __hasProp = {}.hasOwnProperty,
  __extends = function(child, parent) { for (var key in parent) { if (__hasProp.call(parent, key)) child[key] = parent[key]; } function ctor() { this.constructor = child; } ctor.prototype = parent.prototype; child.prototype = new ctor(); child.__super__ = parent.prototype; return child; };

Animal = (function() {
  function Animal(name) {
    this.name = name;
  }

  Animal.prototype.move = function(meters) {
    return alert(this.name + (" moved " + meters + "m."));
  };

  return Animal;

})();

Snake = (function(_super) {
  __extends(Snake, _super);

  function Snake() {
    _ref = Snake.__super__.constructor.apply(this, arguments);
    return _ref;
  }

  Snake.prototype.move = function() {
    alert("Slithering...");
    return Snake.__super__.move.call(this, 5);
  };

  return Snake;

})(Animal);

Horse = (function(_super) {
  __extends(Horse, _super);

  function Horse() {
    _ref1 = Horse.__super__.constructor.apply(this, arguments);
    return _ref1;
  }

  Horse.prototype.move = function() {
    alert("Galloping...");
    return Horse.__super__.move.call(this, 45);
  };

  return Horse;

})(Animal);

sam = new Snake("Sammy the Python");

tom = new Horse("Tommy the Palomino");

sam.move();

tom.move();

\end{verbatim}
\end{tabular}

  If structuring your prototypes classically isn't your cup of tea, CoffeeScript provides a couple of lower-level conveniences. The extends operator helps with proper prototype setup, and can be used to create an inheritance chain between any pair of constructor functions; :: gives you quick access to an object's prototype; and super() is converted into a call against the immediate ancestor's method of the same name. 


\begin{tabular}{p{0.5\textwidth} p{0.5\textwidth}}
\begin{verbatim}
String::dasherize = ->
  this.replace /_/g, "-"


\end{verbatim}
&
\begin{verbatim}
String.prototype.dasherize = function() {
  return this.replace(/_/g, "-");
};

\end{verbatim}
\end{tabular}

  Finally, class definitions are blocks of executable code, which make for interesting metaprogramming possibilities. Because in the context of a class definition, this is the class object itself (the constructor function), you can assign static properties by using \\ 
@property: value, and call functions defined in parent classes: @attr 'title', type: 'text'


\section{Destructuring Assignment}
 To make extracting values from complex arrays and objects more convenient, CoffeeScript implements ECMAScript Harmony's proposed destructuring assignment syntax. When you assign an array or object literal to a value, CoffeeScript breaks up and matches both sides against each other, assigning the values on the right to the variables on the left. In the simplest case, it can be used for parallel assignment: 


\begin{tabular}{p{0.5\textwidth} p{0.5\textwidth}}
\begin{verbatim}
theBait   = 1000
theSwitch = 0

[theBait, theSwitch] = [theSwitch, theBait]

\end{verbatim}
&
\begin{verbatim}
var theBait, theSwitch, _ref;

theBait = 1000;

theSwitch = 0;

_ref = [theSwitch, theBait], theBait = _ref[0], theSwitch = _ref[1];

\end{verbatim}
\end{tabular}

  But it's also helpful for dealing with functions that return multiple values. 


\begin{tabular}{p{0.5\textwidth} p{0.5\textwidth}}
\begin{verbatim}
weatherReport = (location) ->
  # Make an Ajax request to fetch the weather...
  [location, 72, "Mostly Sunny"]

[city, temp, forecast] = weatherReport "Berkeley, CA"

\end{verbatim}
&
\begin{verbatim}
var city, forecast, temp, weatherReport, _ref;

weatherReport = function(location) {
  return [location, 72, "Mostly Sunny"];
};

_ref = weatherReport("Berkeley, CA"), city = _ref[0], temp = _ref[1], forecast = _ref[2];

\end{verbatim}
\end{tabular}

  Destructuring assignment can be used with any depth of array and object nesting, to help pull out deeply nested properties. 


\begin{tabular}{p{0.5\textwidth} p{0.5\textwidth}}
\begin{verbatim}
futurists =
  sculptor: "Umberto Boccioni"
  painter:  "Vladimir Burliuk"
  poet:
    name:   "F.T. Marinetti"
    address: [
      "Via Roma 42R"
      "Bellagio, Italy 22021"
    ]

{poet: {name, address: [street, city]}} = futurists




\end{verbatim}
&
\begin{verbatim}
var city, futurists, name, street, _ref, _ref1;

futurists = {
  sculptor: "Umberto Boccioni",
  painter: "Vladimir Burliuk",
  poet: {
    name: "F.T. Marinetti",
    address: ["Via Roma 42R", "Bellagio, Italy 22021"]
  }
};

_ref = futurists.poet, name = _ref.name, (_ref1 = _ref.address, street = _ref1[0], city = _ref1[1]);

\end{verbatim}
\end{tabular}

  Destructuring assignment can even be combined with splats. 


\begin{tabular}{p{0.5\textwidth} p{0.5\textwidth}}
\begin{verbatim}
tag = "<impossible>"

[open, contents..., close] = tag.split("")

\end{verbatim}
&
\begin{verbatim}
var close, contents, open, tag, _i, _ref,
  __slice = [].slice;

tag = "<impossible>";

_ref = tag.split(""), open = _ref[0], contents = 3 <= _ref.length ? __slice.call(_ref, 1, _i = _ref.length - 1) : (_i = 1, []), close = _ref[_i++];

\end{verbatim}
\end{tabular}

  Destructuring assignment is also useful when combined with class constructors to assign propeties to your instance from an options object passed to the constructor. 
  
  
\begin{tabular}{p{0.5\textwidth} p{0.5\textwidth}}  
\begin{verbatim}
class Person
  constructor: (options) -> 
    {@name, @age, @height} = options


\end{verbatim}
&
\begin{verbatim}
var Person;

Person = (function() {
  function Person(options) {
    this.name = options.name, this.age = options.age, this.height = options.height;
  }

  return Person;

})();

\end{verbatim}
\end{tabular}

\section{Function binding}
 In JavaScript, the this keyword is dynamically scoped to mean the object that the current function is attached to. If you pass a function as a callback or attach it to a different object, the original value of this will be lost. If you're not familiar with this behavior, this Digital Web article gives a good overview of the quirks. 


  The fat arrow =$>$ can be used to both define a function, and to bind it to the current value of this, right on the spot. This is helpful when using callback-based libraries like Prototype or jQuery, for creating iterator functions to pass to each, or event-handler functions to use with bind. Functions created with the fat arrow are able to access properties of the this where they're defined. 


\begin{tabular}{p{0.5\textwidth} p{0.5\textwidth}}
\begin{verbatim}
Account = (customer, cart) ->
  @customer = customer
  @cart = cart

  $('.shopping_cart').bind 'click', (event) =>
    @customer.purchase @cart

\end{verbatim}
&
\begin{verbatim}
var Account;

Account = function(customer, cart) {
  var _this = this;

  this.customer = customer;
  this.cart = cart;
  return $('.shopping_cart').bind('click', function(event) {
    return _this.customer.purchase(_this.cart);
  });
};

\end{verbatim}
\end{tabular}

  If we had used -$>$ in the callback above, @customer would have referred to the undefined ``customer'' property of the DOM element, and trying to call purchase() on it would have raised an exception. 


  When used in a class definition, methods declared with the fat arrow will be automatically bound to each instance of the class when the instance is constructed. 


\section{Embedded JavaScript}
 Hopefully, you'll never need to use it, but if you ever need to intersperse snippets of JavaScript within your CoffeeScript, you can use backticks to pass it straight through.
 
 
\begin{tabular}{p{0.5\textwidth} p{0.5\textwidth}}  
\begin{verbatim}
hi = `function() {
  return [document.title, "Hello JavaScript"].join(": ");
}`

\end{verbatim}
&
\begin{verbatim}
var hi;

hi = function() {
  return [document.title, "Hello JavaScript"].join(": ");
};

\end{verbatim}
\end{tabular}

\section{Switch/When/Else}
 \textbf{Switch}
 statements in JavaScript are a bit awkward. You need to remember to \textbf{break}
 at the end of every \textbf{case}
 statement to avoid accidentally falling through to the default case. CoffeeScript prevents accidental fall-through, and can convert the switch into a returnable, assignable expression. The format is: switch condition, when clauses, else the default case. 


  As in Ruby, \textbf{switch}
 statements in CoffeeScript can take multiple values for each \textbf{when}
 clause. If any of the values match, the clause runs. 


\begin{tabular}{p{0.5\textwidth} p{0.5\textwidth}}
\begin{verbatim}
switch day
  when "Mon" then go work
  when "Tue" then go relax
  when "Thu" then go iceFishing
  when "Fri", "Sat"
    if day is bingoDay
      go bingo
      go dancing
  when "Sun" then go church
  else go work

\end{verbatim}
&
\begin{verbatim}
switch (day) {
  case "Mon":
    go(work);
    break;
  case "Tue":
    go(relax);
    break;
  case "Thu":
    go(iceFishing);
    break;
  case "Fri":
  case "Sat":
    if (day === bingoDay) {
      go(bingo);
      go(dancing);
    }
    break;
  case "Sun":
    go(church);
    break;
  default:
    go(work);
}

\end{verbatim}
\end{tabular}

  Switch statements can also be used without a control expression, turning them in to a cleaner alternative to if/else chains. 


\begin{tabular}{p{0.5\textwidth} p{0.5\textwidth}}
\begin{verbatim}
score = 76
grade = switch
  when score < 60 then 'F'
  when score < 70 then 'D'
  when score < 80 then 'C'
  when score < 90 then 'B'
  else 'A'
# grade == 'C'

\end{verbatim}
&
\begin{verbatim}
var grade, score;

score = 76;

grade = (function() {
  switch (false) {
    case !(score < 60):
      return 'F';
    case !(score < 70):
      return 'D';
    case !(score < 80):
      return 'C';
    case !(score < 90):
      return 'B';
    default:
      return 'A';
  }
})();

\end{verbatim}
\end{tabular}

\section{Try/Catch/Finally}
 Try/catch statements are just about the same as JavaScript (although they work as expressions). 


\begin{tabular}{p{0.5\textwidth} p{0.5\textwidth}}
\begin{verbatim}
try
  allHellBreaksLoose()
  catsAndDogsLivingTogether()
catch error
  print error
finally
  cleanUp()


\end{verbatim}
&
\begin{verbatim}
var error;

try {
  allHellBreaksLoose();
  catsAndDogsLivingTogether();
} catch (_error) {
  error = _error;
  print(error);
} finally {
  cleanUp();
}

\end{verbatim}
\end{tabular}


\section{Chained Comparisons}
 CoffeeScript borrows chained comparisons from Python \^a�� making it easy to test if a value falls within a certain range. 


\begin{tabular}{p{0.5\textwidth} p{0.5\textwidth}}
\begin{verbatim}
cholesterol = 127

healthy = 200 > cholesterol > 60



\end{verbatim}
&
\begin{verbatim}
var cholesterol, healthy;

cholesterol = 127;

healthy = (200 > cholesterol && cholesterol > 60);

\end{verbatim}
\end{tabular}

\section{String Interpolation, Block Strings, and Block Comments}
 Ruby-style string interpolation is included in CoffeeScript. Double-quoted strings allow for interpolated values, using \#\{ ... \}, and single-quoted strings are literal. 
 
 
\begin{tabular}{p{0.5\textwidth} p{0.5\textwidth}} 
\begin{verbatim}
author = "Wittgenstein"
quote  = "A picture is a fact. -- #{ author }"

sentence = "#{ 22 / 7 } is a decent approximation of π"

\end{verbatim}
&
\begin{verbatim}
var author, quote, sentence;

author = "Wittgenstein";

quote = "A picture is a fact. -- " + author;

sentence = "" + (22 / 7) + " is a decent approximation of π";

\end{verbatim}
\end{tabular}

  Multiline strings are allowed in CoffeeScript. 
  
  
\begin{tabular}{p{0.5\textwidth} p{0.5\textwidth}}   
\begin{verbatim}
mobyDick = "Call me Ishmael. Some years ago --
 never mind how long precisely -- having little
 or no money in my purse, and nothing particular
 to interest me on shore, I thought I would sail
 about a little and see the watery part of the
 world..."



\end{verbatim}
&
\begin{verbatim}
var mobyDick;

mobyDick = "Call me Ishmael. Some years ago -- never mind how long precisely -- having little or no money in my purse, and nothing particular to interest me on shore, I thought I would sail about a little and see the watery part of the world...";

\end{verbatim}
\end{tabular}


  Block strings can be used to hold formatted or indentation-sensitive text (or, if you just don't feel like escaping quotes and apostrophes). The indentation level that begins the block is maintained throughout, so you can keep it all aligned with the body of your code. 


\begin{tabular}{p{0.5\textwidth} p{0.5\textwidth}} 
\begin{verbatim}
html = """
       <strong>
         cup of coffeescript
       </strong>
       """
       

\end{verbatim}
&
\begin{verbatim}
var html;

html = "<strong>\n  cup of coffeescript\n</strong>";

\end{verbatim}
\end{tabular}

  Double-quoted block strings, like other double-quoted strings, allow interpolation. 


  Sometimes you'd like to pass a block comment through to the generated JavaScript. For example, when you need to embed a licensing header at the top of a file. Block comments, which mirror the syntax for block strings, are preserved in the generated code. 


\begin{tabular}{p{0.5\textwidth} p{0.5\textwidth}} 
\begin{verbatim}
###
SkinnyMochaHalfCaffScript Compiler v1.0
Released under the MIT License
###



\end{verbatim}
&
\begin{verbatim}
/*
SkinnyMochaHalfCaffScript Compiler v1.0
Released under the MIT License
*/


\end{verbatim}
\end{tabular}

\section{Block Regular Expressions}
 Similar to block strings and comments, CoffeeScript supports block regexes \^a�� extended regular expressions that ignore internal whitespace and can contain comments and interpolation. Modeled after Perl's /x modifier, CoffeeScript's block regexes are delimited by /// and go a long way towards making complex regular expressions readable. To quote from the CoffeeScript source: 


\begin{tabular}{p{0.5\textwidth} p{0.5\textwidth}} 
\begin{verbatim}
OPERATOR = /// ^ (
  ?: [-=]>             # function
   | [-+*/%<>&|^!?=]=  # compound assign / compare
   | >>>=?             # zero-fill right shift
   | ([-+:])\1         # doubles
   | ([&|<>])\2=?      # logic / shift
   | \?\.              # soak access
   | \.{2,3}           # range or splat
) ///



\end{verbatim}
&
\begin{verbatim}
var OPERATOR;

OPERATOR = /^(?:[-=]>|[-+*\/%<>&|^!?=]=|>>>=?|([-+:])\1|([&|<>])\2=?|\?\.|\.{2,3})/;

\end{verbatim}
\end{tabular}

 
\section{ Cake, and Cakefiles }


  CoffeeScript includes a (very) simple build system similar to Make and Rake. Naturally, it's called Cake, and is used for the tasks that build and test the CoffeeScript language itself. Tasks are defined in a file named Cakefile, and can be invoked by running cake [task] from within the directory. To print a list of all the tasks and options, just type cake. 


  Task definitions are written in CoffeeScript, so you can put arbitrary code in your Cakefile. Define a task with a name, a long description, and the function to invoke when the task is run. If your task takes a command-line option, you can define the option with short and long flags, and it will be made available in the options object. Here's a task that uses the Node.js API to rebuild CoffeeScript's parser: 


\begin{tabular}{p{0.5\textwidth} p{0.5\textwidth}} 
\begin{verbatim}
fs = require 'fs'

option '-o', '--output [DIR]', 'directory for compiled code'

task 'build:parser', 'rebuild the Jison parser', (options) ->
  require 'jison'
  code = require('./lib/grammar').parser.generate()
  dir  = options.output or 'lib'
  fs.writeFile "#{dir}/parser.js", code

\end{verbatim}
&
\begin{verbatim}
var fs;

fs = require('fs');

option('-o', '--output [DIR]', 'directory for compiled code');

task('build:parser', 'rebuild the Jison parser', function(options) {
  var code, dir;

  require('jison');
  code = require('./lib/grammar').parser.generate();
  dir = options.output || 'lib';
  return fs.writeFile("" + dir + "/parser.js", code);
});

\end{verbatim}
\end{tabular}

  If you need to invoke one task before another \^a�� for example, running build before test, you can use the invoke function: invoke 'build'. Cake tasks are a minimal way to expose your CoffeeScript functions to the command line, so don't expect any fanciness built-in. If you need dependencies, or async callbacks, it's best to put them in your code itself \^a�� not the cake task. 

\section{ Source Maps }


  CoffeeScript 1.6.1 and above include support for generating source maps, a way to tell your JavaScript engine what part of your CoffeeScript program matches up with the code being evaluated. Browsers that support it can automatically use source maps to show your original source code in the debugger. To generate source maps alongside your JavaScript files, pass the --map or -m flag to the compiler. 


  For a full introduction to source maps, how they work, and how to hook them up in your browser, read the HTML5 Tutorial. 
\section{ ``text/coffeescript'' Script Tags }


  While it's not recommended for serious use, CoffeeScripts may be included directly within the browser using $<$script type=''text/coffeescript''$>$ tags. The source includes a compressed and minified version of the compiler (Download current version here, 39k when gzipped) as extras/coffee-script.js. Include this file on a page with inline CoffeeScript tags, and it will compile and evaluate them in order. 


  In fact, the little bit of glue script that runs ``Try CoffeeScript'' above, as well as the jQuery for the menu, is implemented in just this way. View source and look at the bottom of the page to see the example. Including the script also gives you access to CoffeeScript.compile() so you can pop open Firebug and try compiling some strings. 


  The usual caveats about CoffeeScript apply \^a�� your inline scripts will run within a closure wrapper, so if you want to expose global variables or functions, attach them to the window object. 
\section{ Books }


  There are a number of excellent resources to help you get started with CoffeeScript, some of which are freely available online. 
\begin{itemize}
\item The Little Book on CoffeeScript is a brief 5-section introduction to CoffeeScript, written with great clarity and precision by Alex MacCaw. 
\item Smooth CoffeeScript is a reimagination of the excellent book Eloquent JavaScript, as if it had been written in CoffeeScript instead. Covers language features as well as the functional and object oriented programming styles. By E. Hoigaard. 
\item CoffeeScript: Accelerated JavaScript Development is Trevor Burnham's thorough introduction to the language. By the end of the book, you'll have built a fast-paced multiplayer word game, writing both the client-side and Node.js portions in CoffeeScript. 
\item CoffeeScript Programming with jQuery, Rails, and Node.js is a new book by Michael Erasmus that covers CoffeeScript with an eye towards real-world usage both in the browser (jQuery) and on the server size (Rails, Node). 
\item CoffeeScript Ristretto is a deep dive into CoffeeScript's semantics from simple functions up through closures, higher-order functions, objects, classes, combinators, and decorators. By Reg Braithwaite. 
\item Testing with CoffeeScript is a succinct and freely downloadable guide to building testable applications with CoffeeScript and Jasmine. 

\end{itemize}
\section{ Screencasts }
\begin{itemize}
\item A Sip of CoffeeScript is a Code School Course which combines 6 screencasts with in-browser coding to make learning fun. The first level is free to try out. 
\item Meet CoffeeScript is a 75-minute long screencast by PeepCode. Highly memorable for its animations which demonstrate transforming CoffeeScript into the equivalent JS. 
\item  If you're looking for less of a time commitment, RailsCasts' CoffeeScript Basics should have you covered, hitting all of the important notes about CoffeeScript in 11 minutes. 

\end{itemize}
\section{ Examples }


  The best list of open-source CoffeeScript examples can be found on GitHub. But just to throw out few more: 
\begin{itemize}
\item \textbf{github}
's Hubot, a friendly IRC robot that can perform any number of useful and useless tasks. 
\item \textbf{sstephenson}
's Pow, a zero-configuration Rack server, with comprehensive annotated source. 
\item \textbf{technoweenie}
's Coffee-Resque, a port of Resque for Node.js. 
\item \textbf{assaf}
's Zombie.js, a headless, full-stack, faux-browser testing library for Node.js. 
\item \textbf{jashkenas}
' Underscore.coffee, a port of the Underscore.js library of helper functions. 
\item \textbf{stephank}
's Orona, a remake of the Bolo tank game for modern browsers. 
\item \textbf{josh}
's nack, a Node.js-powered Rack server. 

\end{itemize}
\section{ Resources }
\begin{itemize}
\item Source Code\\ 
 Use bin/coffee to test your changes,\\ 
bin/cake test to run the test suite,\\ 
bin/cake build to rebuild the CoffeeScript compiler, and \\ 
bin/cake build:parser to regenerate the Jison parser if you're working on the grammar. \\ 
\\ 
git checkout lib \&\& bin/cake build:full is a good command to run when you're working on the core language. It'll refresh the lib directory (in case you broke something), build your altered compiler, use that to rebuild itself (a good sanity test) and then run all of the tests. If they pass, there's a good chance you've made a successful change. 
\item CoffeeScript Issues\\ 
 Bug reports, feature proposals, and ideas for changes to the language belong here. 
\item CoffeeScript Google Group\\ 
 If you'd like to ask a question, the mailing list is a good place to get help. 
\item The CoffeeScript Wiki\\ 
 If you've ever learned a neat CoffeeScript tip or trick, or ran into a gotcha \^a�� share it on the wiki. The wiki also serves as a directory of handy text editor extensions, web framework plugins, and general CoffeeScript build tools. 
\item The FAQ\\ 
 Perhaps your CoffeeScript-related question has been asked before. Check the FAQ first. 
\item JS2Coffee\\ 
 Is a very well done reverse JavaScript-to-CoffeeScript compiler. It's not going to be perfect (infer what your JavaScript classes are, when you need bound functions, and so on...) \^a�� but it's a great starting point for converting simple scripts. 
\item High-Rez Logo\\ 
 The CoffeeScript logo is available in Illustrator, EPS and PSD formats, for use in presentations. 

\end{itemize}
\section{ Web Chat (IRC) }


  Quick help and advice can usually be found in the CoffeeScript IRC room. Join \#coffeescript on irc.freenode.net, or click the button below to open a webchat session on this page. 

\end{document}
